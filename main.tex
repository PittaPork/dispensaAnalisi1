\documentclass{article}
\usepackage[most]{tcolorbox}
\usepackage{xcolor}
\usepackage{graphicx, float}
\usepackage[T1]{fontenc}
\usepackage[utf8]{inputenc}
\usepackage[sfdefault]{atkinson}
\graphicspath{{images/}}
\usepackage{amsthm, amssymb, amsmath}
\usepackage[a4paper]{geometry}
\usepackage[dvipsnames]{xcolor}
\usepackage[skip=8pt, indent= 0pt]{parskip}
 \geometry{
 a4paper,
 total={170mm,257mm},
 left=20mm,
 top=20mm,
 }
\title{Analisi Matematica}
\author{Marco Pittarello}
\date{}

\newtcbtheorem{theo}{Teorema}{%
    colframe=blue!80!white,
    colback=blue!10!white
}{th}

\newtcbtheorem{ex}{Esempio}{%
    colframe=OliveGreen!80!white,
    colback=OliveGreen!5!white
}{ex}

\newtcbtheorem{defi}{Definizione}{%
    colframe=BrickRed!80!white,
    colback=Red!5!white
}{def}

\begin{document}
\maketitle
\tableofcontents
\newpage
\section{Principio di Induzione}

\begin{defi}{}{}
Il principio di induzione è un metodo per dimostrare predicati matematici.
\end{defi}
come \\
\begin{itemize}
    \item []$\forall n \in \mathbb{N} \qquad \underbrace{1 + 2 + 3 + \dots + n \ = \ \frac{n\ (n+1)}{2}}_{\text{P}(n)}$\\
    \item []$\forall n \in \mathbb{N} \qquad \underbrace{\forall x \in \mathbb{R} \quad x >  1 \quad (1+x)^n \ge nx+1}_{\text{P}(n)}$
\end{itemize}
\begin{theo}{1° forma}{}
Sia P$(n)$ un predicato con parametro $n \in \mathbb{N}$ e tale che:
\begin{enumerate}
    \item P$(0)$ è vero (\colorbox{yellow}{caso base})
    \item $\forall n \in \mathbb{N}\quad \text{P}(n) \xrightarrow{}\text{P}(n+1)$ (\colorbox{yellow}{passo induttivo})
\end{enumerate}
Allora $P(n)$ è vera $\forall n \in \mathbb{N}$
\end{theo}\ 

\begin{ex}{}{}
Dimostrare che $\forall n \in \mathbb{N} \quad \underbrace{2^n \ge n+1}_{\text{P}(n)}$.
\end{ex}
\underline{CASO BASE} : \quad P$(0) \quad 2^0 \ge 1 \quad \text{vero}$

\underline{PASSO INDUTTIVO} : \quad  $\forall n \in \mathbb{N} \quad \text{P}(n) \xrightarrow{} \text{P}(n+1)$\\

\begin{itemize}
    \item[] Suppongo che $2^n \ge n+1$ e dimostro che $2^{n+1} \ge n +2$
    \item[] $2^n \ge n+1\ \xrightarrow{}\ 2\cdot 2^n \ge 2 \cdot (n+1) $
    \item[] $2^{n+1}\ \ge\ 2n+2\ge n+2$
    \item[] Dunque abbiamo dimostrato che P$(n)\xrightarrow{}\text{P}(n+1)$
    \item[] Dunque per il principio di induzione è vero che $\forall n \in \mathbb{N}$ vale P$(n)$\\
\end{itemize}

\begin{ex}{}{}
    Dimostrare che $\forall n \in \mathbb{N}$ : \[\sum^n_{k=0}k = \frac{n(n+1)}{2}\]
\end{ex}
\underline{CASO BASE} : P$(0):\ $"$0$ = $0$" è vera

\underline{PASSO INDUTTIVO} :\quad Assumo che P$(n)$ è vera e dimostro che è vera anche P$(n+1)$

    \[\text{ovvero}\qquad \sum^{n+1}_{k=0}k=\frac{(n+1)(n+2)}{2}\]

    \[\sum^{n+1}_{k=0}k=\ \sum^{n}_{k=0}k+(n+1)\ =\ \frac{n(n+1)}{2}+(n+1)=\ \frac{n(n+1)+2(n+1)}{2}=\ \frac{(n+2)(n+1)}{2}\]
Ho dimostrato CASO BASE e PASSO INDUTTIVO, dunque per il principio di induzione segue che: 
\[\forall n\in\mathbb{N}\quad \text{P}(n)\]
\begin{theo}{2° forma}{}
    Sia P$(n)$, $n \in\mathbb{N}$\quad un predicato tale che:
    \begin{enumerate}
        \item P$(0)$ è vera (\colorbox{yellow}{caso base})
        \item $\forall n\in\mathbb{N}$,\quad $n \ge 1$ (\colorbox{yellow}{passo induttivo})
    \end{enumerate}
    Se $\forall m\in\mathbb{N}\quad 0 \le m \le n\quad \text{P}(n)$ è vera allora lo è anche P$(m)$ (\colorbox{yellow}{ipotesi induttiva})\\Allora $\forall n\in\mathbb{N}\quad \text{P}(n)\quad $ è vera
\end{theo}
\colorbox{yellow}{OSSERVAZIONE} è una forma "più forte" della 1° forma, poichè per dimostrare P$(n)$ si usa la condizione che P$(m)$ vale per tutti gli $m < n$\\\\
\colorbox{yellow}{OSSERVAZIONE} in entrambe le forme del principio di induzione possiamo sostituire $0$ con qualunque $n_0\in\mathbb{N}$. Ovvero, se per un predicato P$(m)$ dimostriamo:
\begin{itemize}
    \item il caso base per $n_o$
    \item il passo induttivo $\forall n\ge n_0$
\end{itemize}
Allora possiamo concludere che $\forall n\ge n_0 \text{P}(n)$ è vera

\begin{ex}{}{}
    Dimostriamo che $\forall n\in\mathbb{N}\quad n\ge2\quad \underbrace{n\text{ si può scrivere come prodotto di numeri primi}}_{\text{P}(n)}$
\end{ex}
\underline{CASO BASE} : P$(2)$ è banalmente vera: 2 è un numero primo


\underline{PASSO INDUTTIVO} : dimostriamo che $\forall n\in\mathbb{N} \quad n\ge3\quad (\forall\quad1\le m<n\quad\text{P}(m)) \xrightarrow{}\text{P}(n)$
\begin{itemize}
    \item[] Ovvero, assumendo che P$(m)$ vale\quad$\forall\quad1\le m<n$, ovvero si può scrivere come prodotto di primi, dimostriamo che anche $n$ si scrive come prodotto di primi ci sono due casi.
    \item[] Se $n$ è primo allore è chiaramente prodotto di primi
    \item[] Se $n$ non è primo allora è divisibile per un numero $m_1$ con $m_1 \ne n$ e $m_1\ne1$, 
    \item[] in particolare $2\le m_1<n$
    \item[] Dunque $\exists m_2\in\mathbb{N}$ tale che
    \item[] $n=m_1m_2$ con $m_1$,$m_2$ diversi da $m$ e da $1$ 
    \item[] Inoltre $2\le m_2<n$
    \item[] perchè $m_1<n$\qquad$m_1>1$
    \item[] Per l'ipotesi induttiva P$(m_1)$ e P$(m_2)$ sono vere.
\end{itemize}
\section{Coefficenti Binomiali}
\begin{defi}{}{}
    Definiamo C$_{n,k}$ = numero totale di modi possibile, e si chiama:
\begin{itemize}
    \item[] \underline{numero di combinazioni di $n$ elementi di classe $k$} 
\end{itemize}
\vspace{5pt}Spesso C$_{n,k}$ viene anche denotato con il simbolo $\binom{n}{k}$, chiamato
\begin{itemize}
    \item[] \underline{coefficiente binomiale $n$ su $k$}
\end{itemize}
\end{defi}
\colorbox{yellow}{Quanto vale $\binom{n}{k}$?}

\[\forall n\in\mathbb{N}\quad \forall k\in\mathbb{N}\quad k\le n\qquad\binom{n}{k} = C_{n,k} =\frac{n!}{k!\ (n-k)!}\]
\subsection{Proprietà di $\binom{n}{k}$}
    \begin{theo}{}{}
        $\forall n\in\mathbb{N}\quad\forall k\in\mathbb{N}\quad1\le k\le n$ si ha:\\
        \begin{itemize}
            \item [] $\binom{n+1}{k}=\ \binom{n}{k}+\binom{n}{k-1}$
        \end{itemize}
    \end{theo}
\colorbox{yellow}{OSSERVAZZIONE:} Abbiamo un altro metodo per calcolare $\binom{n}{k}\forall n\in\mathbb{N}\quad\forall k\le n$\\
Utilizzando il teorema e $\binom{n}{0}=1$ possiamo calcolare ogni valore di $\binom{n}{k}\quad\forall n\in\mathbb{N}\quad\forall k\le n$, evitando di dover calcolare ogni volta i fattoriali.
\begin{ex}{Triangolo di Tartaglia}{}
    Si costruisce elencando per righe i coefficienti binomiali, la riga $n$ è composta da $\binom{n}{0},\binom{n}{1},\dots,\binom{n}{m}$
\end{ex}
\begin{gather}
    \binom{0}{0} = 1\\
    \binom{1}{0} = 1\qquad\binom{1}{1} = 1\\
    \binom{2}{0}=1\qquad\binom{2}{1}=\ \binom{1}{0}+\binom{1}{2}=2\qquad\binom{2}{2}=1\\
    \binom{3}{0}=1\qquad\qquad\binom{3}{1}=3\qquad\qquad\quad\qquad\binom{3}{2}=3\qquad\qquad\binom{3}{0}=1
\end{gather}

\end{document}